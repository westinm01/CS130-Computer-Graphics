\documentclass[12pt]{article}

\usepackage[pdf]{graphviz}

\setlength{\parindent}{0em}
\setlength{\parskip}{.5em}

\usepackage{framed}
\newcounter{problem}
\newcounter{problempart}[problem]
\newcounter{solutionpart}[problem]
\newenvironment{problem}{\stepcounter{problem}\noindent{\bf\arabic{problem}.}}{\setcounter{problempart}{0}\setcounter{solutionpart}{0}}
\newenvironment{solution}{\par\textcolor{green!50!black}\bgroup}{\egroup\par}
\newcommand{\qpart}{\stepcounter{problempart}${}$\\\noindent{(\alph{problempart})} }
\newcommand{\spart}{\stepcounter{solutionpart}${}$\\\noindent{(\alph{solutionpart})} }
\newcommand{\TODO}{\textcolor{red}{$\blacksquare$}}
\newcommand{\SOL}[1]{\textcolor{green!50!black}{#1}}

\usepackage{hyperref}
\usepackage{fullpage}
\usepackage{amsmath,mathabx,MnSymbol}
\usepackage{color,tikz}
\usepackage{pstricks}
\usepackage{pst-plot,pst-node}
\usepackage{footnote,enumitem}
\usepackage{longtable}
\newcommand{\mx}[1]{\begin{pmatrix}#1\end{pmatrix}}
\definecolor{dkgreen}{rgb}{0,.5,0}
\usepackage{algorithm}
\usepackage[noend]{algpseudocode}

\newcommand{\uu}{\mathbf{u}}
\newcommand{\vv}{\mathbf{v}}
\newcommand{\cc}{\mathbf{c}}
\newcommand{\ww}{\mathbf{w}}
\newcommand{\xx}{\mathbf{x}}
\newcommand{\zz}{\mathbf{z}}
\newcommand{\ee}{\mathbf{e}}
\newcommand{\pp}{\mathbf{p}}
\newcommand{\qq}{\mathbf{q}}
\renewcommand{\AA}{\mathbf{A}}
\newcommand{\BB}{\mathbf{B}}
\newcommand{\nn}{\mathbf{n}}
\newcommand{\gp}[1]{\left(#1\right)}

\newcommand{\TODOL}[1]{\textcolor{red}{\underline{\hspace{#1 cm}}}}

\usepackage{listings}

\lstset{
  language=C++,
  showstringspaces=false,
  identifierstyle=\color{magenta},
  basicstyle=\color{magenta},
  keywordstyle=\color{blue},
  identifierstyle=\color{black},
  commentstyle=\color{green},
  stringstyle=\color{red}
}

\begin{document}

\title{CS130 - LAB - Introduction to OpenGL}
\date{}
\author{Name: \TODOL7\qquad\qquad SID: \TODOL4}
\maketitle
\begin{center}
\end{center}

\section*{Introduction}

Open Graphics Library (OpenGL) is a cross-platform API for fast rendering of 2D
and 3D graphics. OpenGL typically runs on a graphics processing unit (GPU) and
it is optimized to render multiple images per second. For this reason, OpenGL is
often used in game engines and other applications that require interactivity
with the user.

The goal of this lab is to get you started with OpenGL by implementing Phong's
illumination model into special OpenGL programs called shaders.

The process is summarized as follows:

\begin{itemize}
\item An OpenGL program is written in C/C++ and consists of setting up the scene
  (camera position, objects, lights, among others).
\item The OpenGL program is also responsible for reading a text file with shader
  code, compiling it and sending it to the GPU for execution.
\item The language used in the shader program is very similar to C and is called
  OpenGL shader language (GLSL)
\item The shader typically runs on the GPU and the shader determines the
  position and color of vertices. Vertices are the points that constitute
  geometry. For instance, a cube has 8 vertices.
\item Here, we care about two types of shaders: vertex and fragment.
\item The vertex shader receives vertices and applies transformations to these
  vertices (scale, translation, rotation, among others).
\item The fragment shader receives fragments and determines the color of that
  fragment. Fragments are transformed vertices outputted by the vertex shader
  after rasterization.
\end{itemize}

The left diagram below depicts the process of loading the vertex and fragment
shaders in the OpenGL C/C++ code. The right diagram depicts the vertex and
fragment shaders.  Taken from\\
\href{https://www.google.com/url?q=http://www.lighthouse3d.com/tutorials/glsl-12-tutorial/pipeline-overview/\&sa=D\&ust=1601679253988000\&usg=AOvVaw24XGbOqgiEH0AFlwC_2n_a}{http://www.lighthouse3d.com/tutorials/glsl-12-tutorial/pipeline-overview/}.

\includegraphics[width=.5\textwidth]{images/gpu-diagram.png}%
\includegraphics[width=.5\textwidth]{images/gpu-flow.png}

1. Consider the OpenGL code diagram depicted above. Describe briefly with your
own words each one of the following functions. Look at the OpenGL documentation
for reference.  Google: ``opengl 4 references.''

Link:
\href{https://www.google.com/url?q=https://www.khronos.org/registry/OpenGL-Refpages/gl4/\&sa=D\&ust=1601679253989000\&usg=AOvVaw04dktDekd6O2yRGd862SNn}{https://www.khronos.org/registry/OpenGL-Refpages/gl4/}

\texttt{glCreateShader}\\
input: \TODOL6\\
output: \TODOL6\\*[1em]
\texttt{glShaderSource}\\
input: \TODOL6\\*[1em]
\texttt{glCompileShader}\\
input: \TODOL6\\*[1em]
\texttt{glCreateProgram}\\
output: \TODOL6\\*[1em]
\texttt{glAttachShader}\\
input: \TODOL6\\*[1em]
\texttt{glLinkProgram}\\
input: \TODOL6\\*[1em]
\texttt{glUseProgram}\\
input: \TODOL6\\*[1em]

2. Read the comments and order the lines of code in correct order for loading
shaders.  Fill in the blanks afterwards.

\begin{enumerate}[label=\Alph*.]
\item glCompileShader(\TODOL2); // compile fragment shader
\item glAttachShader(\TODOL2, \TODOL2); // attach vertex shader to program
\item GLuint vertex\_id = glCreateShader(\TODOL2); // create vertex shader
\item glCompileShader(\TODOL2); // compile vertex shader
\item glAttachShader(\TODOL2, fragment\_id); // attach fragment shader to program
\item glShaderSource(\TODOL2, 1, \&vertex\_shader\_file, NULL); // source vertex shader
\item glLinkProgram(\TODOL2); // link program
\item GLuint fragment\_id=glCreateShader(\TODOL2); // create fragment shader
\item glShaderSource(\TODOL2, 1, \&fragment\_shader\_file, NULL); // source fragment shader
\item GLuint program = glCreateProgram();
\end{enumerate}
Ordering:\TODOL6
\begin{lstlisting}
// vertex shader
void main()
{
  gl_Position = gl_ProjectionMatrix * gl_ModelViewMatrix * gl_Vertex;
  gl_FrontColor = vec4(0, 1, 0, 1);
}
\end{lstlisting}

\begin{lstlisting}
// fragment shader
vec4 light_color = vec4(1, 1, 0, 1);
void main()
{
  gl_FragColor = light_color*gl_FrontColor;
}
\end{lstlisting}

The vertex shader receives a \texttt{vec4 gl\_Vertex} and returns a
\texttt{vec4 gl\_Position}. \texttt{gl\_ProjectionMatrix} and
\texttt{gl\_ModelViewMatrix} are transformation matrices given by OpenGL.

The fragment shader receives \texttt{gl\_FrontColor} from the vertex shader and
returns the color of the fragment as \texttt{gl\_FragColor}.

3.1. What is the output color of the fragment shader?

\texttt{gl\_FragColor = (\TODOL1,\TODOL1,\TODOL1,\TODOL1)}

3.2. Consider an object with color green represented by the RGB color vector (0,
1, 0) and a blue light source with color (0, 0, 1). If we illuminate the object
with the light, what is the output color?

\begin{solution}
  \textbf{\textcolor{red}{\TODO}}
\end{solution}

\section*{Part 2: Phong model}

Write the equations for the the Phong model components. Draw any missing vectors
in the figure below.

Ambient: \TODOL6\\
Diffuse: \TODOL6\\
Specular: \TODOL6

\includegraphics[width=8cm]{images/light-rays.png}

In the figure below, vector $\mathbf{r}$ is the reflection of vector $\mathbf{l}$ from the
surface, and $\mathbf{n}$ vector is the unit-length normal of the surface.

\includegraphics[width=8cm]{images/reflection-direction.jpg}

Write the reflection vector $\mathbf{r}$ in terms of $\mathbf{n}$ and $\mathbf{l}$, following the steps
below:
\begin{enumerate}
\item Formulate vector $\mathbf{p}$, which is the projection of $\mathbf{l}$ on $\mathbf{n}$, in terms of
  $\mathbf{l}$ and $\mathbf{n}$. $\mathbf{p}= \TODOL6$.
\item Formulate vector $\mathbf{d}$, in terms of $\mathbf{l}$ and $\mathbf{p}$.  $\mathbf{d}= \TODOL6$.
\item Write vector $\mathbf{r}$ in terms of $\mathbf{d}$, $\mathbf{p}$ and $\mathbf{l}$ (you do not have to use all
  of them). $\mathbf{r} = \TODOL6$.
\item Substitute $\mathbf{p}$ and $\mathbf{d}$, with your results from steps 1 and 2, and write
  $\mathbf{r}$ in terms of $\mathbf{l}$ and $\mathbf{n}$ only.  $\mathbf{r}=\TODOL6$.
\end{enumerate}

In order to write Phong's model in your shader, you can use (these structures and variables are already defined elsewhere in the program):

\begin{lstlisting}
struct gl_LightSourceParameters
{
  vec4 ambient;
  vec4 diffuse;
  vec4 specular;
  vec4 position;
};
uniform gl_LightSourceParameters
gl_LightSource[gl_MaxLights];
struct gl_LightModelParameters
{
  vec4 ambient;
};
uniform gl_LightModelParameters gl_LightModel;
struct gl_MaterialParameters
{
  vec4 ambient;
  vec4 diffuse;
  vec4 specular;
  float shininess; // this is the exponent of the specular component
};
uniform gl_MaterialParameters gl_FrontMaterial;
\end{lstlisting}

You may also use the following functions: \texttt{max($\cdot$,$\cdot$); dot($\cdot$,$\cdot$);
normalize($\cdot$);}

You can assume the camera position is at the origin, i.e., at coordinates (0, 0,
0).

\section*{Part 3: Notes on Assignment 1 - Checkpoint 2}

If you implemented plane intersection, then you have test 04 working.  The next
steps are:

\begin{enumerate}
\item Phong shader
\item Shadows
\end{enumerate}

Starting with the Phong shader.  (Implement \texttt{Shade\_Surface} in
\texttt{phong\_shader.cpp}). Recall Phong shader consists of 3 components:
ambient, diffuse, specular. You will need to calculate each component and add
them all to the color that is returned.

\paragraph*{Ambient.} Combination of three variables (you have access to all of
them in \texttt{Shade\_Surface})
\begin{enumerate}
\item \texttt{world.ambient\_color}
\item \texttt{world.ambient\_intensity}
\item \texttt{color\_ambient}
\end{enumerate}

\paragraph*{Diffuse.} Is proportional to the cosine of the angle between the
normal ($n$) and the vector from the intersection point to the light source
($l$). This term is the intensity of the diffuse component.
\begin{itemize}
\item The intersection point is calculated as the point in the ray with the
  earliest hit $t$. You can get any point on a ray using the function
  \texttt{ray.Point(t)}.
\item You may need to calculate the intersection point in your
  \texttt{Cast\_Ray} before passing it to the shader.
\item Notice the normal should be pointing to outside of your object. If the
  nearest point is exiting the object, you may need to invert the normal so it
  is facing the right direction.
\item Normalize the vectors when calculating the cosine using dot product.
\item Check if the light source is behind the intersection point on the
  surface. In this case, the diffuse intensity is zero. You can check for this
  by taking $\max(l \cdot n, 0)$.
\item You have access to \texttt{color\_diffuse} in your Phong shader. This
  should be combined with the diffuse intensity.
\item You will also need to compute the color of the light source and combine it
  in your diffuse component. In particular, the intensity of the light should
  decay proportional to the square distance between the intersection point and
  the light source.
\item You can get the light color at the proper intensity by calling the function
  \texttt{Emitted\_Light} passing the vector between the light source and the intersection point.
\end{itemize}

\paragraph*{Specular.} Proportional to the cosine of the angle between the
reflected direction and the vector from the intersection point to the camera
position ($c$).
\begin{itemize}
\item You can calculate the reflected direction using $r=(2*(l \cdot n)
  n-l)$. Make sure $l$ and $r$ are normalized.
\item The specular intensity is $\max(r \cdot c,0)^\alpha$, where $\alpha$ is
  given to you as the \texttt{specular\_power} variable.
\item The final color is calculated similarly to the diffuse component by using
  the \texttt{light\_color} with decay proportional to the square of the
  distance to the light source.
\end{itemize}

\paragraph*{Shadows.}
\begin{itemize}
\item In your Phong shader, check if shadows should be calculated by using the
  variable \texttt{world.enable\_shadows}.
\item If \texttt{world.enable\_shadows} is true, then you should check if there
  is an object between your intersection point and the light source (You can use
  \texttt{Closest\_Intersection} for this).
\item If there is an object blocking all your light sources, then you should
  return only the ambient light component.
\end{itemize}

\end{document}
